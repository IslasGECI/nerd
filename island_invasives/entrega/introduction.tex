\section{Introduction}
The effects of invasive rodent species on island ecosystems are incredibly deleterious, especially on islands that present high levels of endemism and islands that have evolved in the absence of predators occupying similar niches to the invasive rodent species or higher order predators \cite{Meyers2000}. Under these circumstances, the presence of invasive rodents on islands can lead to the rapid decline and extinction of native plant and animal species (\cite{Medina2011} and \cite{Towns2006}). The resultant losses are reflected in reduced biodiversity on the affected islands and in many cases, the emergence of the invasive rodent as the dominant species. In severe cases of rodent invasion, key island ecosystem services are lost (e.g. \cite{Towns2006}). As such, the first step in island restoration and biodiversity recovery is the eradication of invasive rodent species.

Of the various means of rodent eradication on islands, the aerial broadcast of rodenticide bait is one of the preferred methods given the obvious advantages. The aerial dispersal of rodenticide can cover large areas quickly and can mitigate the challenges associated with complex topography. To assess the effectiveness of an aerial operation, bait density maps are required to evaluate the spatial variation of bait availability on the ground. However, creating bait density maps has been traditionally slow and impractical in the field, while taking in situ measurements to evaluate aerial work is difficult given the challenges associated with field conditions, topography, and available manpower.

To address these challenges, we have developed NERD: Numerical Estimation of Rodenticide Dispersal. NERD facilitates the planning of helicopter rodenticide dispersal campaigns by generating bait density maps automatically and allowing for the instant identification of bait gaps with fewer in situ measurements. NERD consists of two components, a mathematical model and its implementation in the computing language of MATLAB. The mathematical model is based on prior calibration experiments in which the mass flow of rodenticide through a bait bucket is measured. At its core, the model is a probability density function that describes bait density as a function of bucket aperture diameter, helicopter speed, and wind speed.
