\section{Resumen}
Los roedores invasores están presentes en aproximadamente el 90\% de las islas del mundo y constituyen una de las amenazas más graves a las especies insulares endémicas y nativas. La erradicación de los roedores es fundamental para los esfuerzos de conservación de las islas y la transmisión aérea de cebo rodenticida es el método de dispersión preferido. Para maximizar la eficacia de las campañas de erradicación de roedores utilizando métodos de dispersión aérea, se necesita la generación de mapas de densidad de cebo precisos y en tiempo real. Tradicionalmente, la creación de mapas para estimar la dispersión espacial del cebo en el suelo se ha llevado a cabo utilizando SIG, que se basa en varias suposiciones no probadas y requiere mucho tiempo. Para mejorar la precisión y acelerar la evaluación de las operaciones aéreas, se desarrolló una nueva herramienta llamada NERD: Estimación Numérica de Dispersión de Rodenticida. NERD es la implementación de un modelo matemático en el lenguaje computacional de MATLAB que realiza cálculos con mayor precisión, mostrando resultados casi en tiempo real. En su núcleo, el modelo es una función de densidad de probabilidad que describe la densidad del cebo como una función del diámetro de la apertura de la cubeta del cebo, de la rapidez del helicóptero, y de la velocidad del viento. NERD también facilita la planificación de trayectorias de vuelo de helicópteros y permite la identificación instantánea de zonas con ausencia de cebo. NERD se utilizó efectivamente en dos campañas recientes y exitosas de erradicación de roedores en México: la erradicación del ratón en la isla San Benito Oeste (400 ha) en el Pacífico mexicano y la erradicación de la rata en la isla Cayo Centro (539 ha) de Banco Chinchorro en el Caribe Mexicano. Esta última representa la mayor erradicación de roedores en una isla tropical húmeda hasta la fecha. NERD ha demostrado su eficacia y puede ahorrar cantidades significativas de dinero en las campañas de erradicación de roedores a gran escala.
